\begin{abstract}
4차 산업혁명이 시작되고, 수많은 데이터가 정보화되어 처리되고 있다. 또, 인공지능과 같은 첨단 프로그램들은 많은 양의 데이터를 빠르게 처리해야 하기에 고성능 컴퓨터를 필요로 한다. 본 설명서는 리눅스 서버를 처음 접하는 정보과학 연구자들을 위해 작성되었다. 특히, 리눅스 서버 이용을 위해 필수적인 ssh, sftp 등의 사용 방법, 공용 서버 사용 규칙, 인공지능 연구를 위해 필수적인 Python, CUDA, TensorFlow, Torch, venv 등의 사용 방법을 다뤘다. 또, 함께 쓰는 공동 연구 환경에서 어떻게 자신의 연구를 해나가야 하는지 방향을 제시했다.~\\


정보과학 연구의 세계는 깊고도 넓기에 연구의 의미와 가치를 잘 이애하고 자신에게 맞는 연구를 하는 것이 중요하다. 연구 방법을 익히고, 다양한 사람들의 연구를 경험해봐야 한다. 저자는 본 설명서를 통해, 서버를 이용한 정보과학 연구에 꼭 필요한 것을들 가려 뽑아 연구의 모든 것을 간명하게 전달하려고 노력했다. 본 설명서가 귀하의 미래를 밝히는 길잡이가 되기를 기원한다.~\\\\
\end{abstract}
\begin{dedication2}
{\large{37기 정보과 일동}}\\[5mm]
2021년 5월 17일, 송죽골에서.
\end{dedication2}