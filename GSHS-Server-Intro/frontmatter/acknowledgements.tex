\begin{acknowledgements}
\textbf{
서버의 효율적인 이용을 위해 아래와 같은 규칙들이 있다. 반드시 숙지하고 준수하기 바란다.}
~\\


\begin{itemize}
 \item 서버는 연구 및 공적 목적으로만 사용 가능하다.
 \item 서버는 함께 사용한다.
 \item 계정명은 반드시 gs00000 (학번)으로 한다.
 \item 계정 생성시 자신의 홈 디렉토리에 whoami.txt를 만들어 자신의 이름, 연락처, 이메일 주소를 기록해야 한다.
 \item GPU는 송죽학사를 통해 신청한 자만이 사용할 수 있다.
 \item GPU 1개는 최장 12일 동안, 2개는 최장 6일 동안, ..., 6개는 최장 2일 동안 사용 가능하다.
 \item GPU를 신청하지 않은 상태에서 사용할 시 프로세스가 강제 종료될 수 있다.
 \item 과도한 CPU/GPU/RAM 사용으로 서버에 과부하를 유발하는 프로세스는 강제로 종료될 수 있으며 사용자는 경고를 받을 수 있다. 경고 3회 누적시 GPU 신청이 제한될 수 있다.
 \item 관리자 권한이 없는 계정은 졸업 시 삭제를 원칙으로 한다.
 \item 관리자 계정은 GM 봉사모둠원만이 소유한다.
 \item 허가 없이 reboot 명령어를 사용해서는 안 된다.
 \item 모든 사용자는 자신의 가상환경을 구성해 사용해야 한다.
\end{itemize}~\\



\textbf{
아래는 유용한 링크들이다.}
~\\
\begin{itemize}
 \item 경기과학고등학교 연구용 서버 사용 및 관리 설명서 (\href{https://www.overleaf.com/project/60a1d2365a26a733e5a9b16f}{Overleaf 주소})
 \item 경기과학고등학교 리눅스 사용자 협회 (\href{https://github.com/gshslinuxintro}{Github 주소})
 \item 경기과학고등학교 리눅스 사용자 협회 오픈채팅방 (\href{https://open.kakao.com/o/gL8MCked}{카카오톡 주소})
 \item 현재 연구용 서버 IP (\href{http://115.23.235.135}{서버 주소})
\end{itemize}~\\
\end{acknowledgements}